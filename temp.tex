\documentclass[tikz]{article}

\usepackage{pgfplots}
\usepgfplotslibrary{groupplots}
%\usepgflibrary{pgfplots.groupplots}
\pgfplotstableread[col sep = comma]{bound_SE.dat}\SEbound
\pgfplotsset{
        my legend style compare/.style={
            legend style={
                at={(2.35,.85)},
                anchor=north west,
            },
            legend columns=1,
	    legend style={font=\small},
        },
        %grid=minor,
            x tick label style={font =\small, /pgf/number format/1000 sep=},
        cycle multi list={
        {red, line width=0.6pt, mark=o,mark size=2pt}, 
        {blue, line width=0.6pt, mark=square,mark size=1.5pt}, 
        {teal, line width=0.6pt, mark=triangle,mark size=1.5pt},
		},
}
\begin{document}

%\begin{figure}[t!]
%\begin{center}
%\begin{tikzpicture}
%    \begin{groupplot}[
%        group style={
%            group size=2 by 1,
%        },
%        width = .5\textwidth,
%        height = 5cm
%    	]
%        \nextgroupplot[
%        		view={0}{90},
%                xlabel=$n$,
%                ylabel=$\alpha$,
%                colormap/blackwhite,
%        ]
%			\addplot3[surf] file {PhaseTransvsn_bounded.dat};
%        
%        \nextgroupplot[
%		        view={0}{90},
%                xlabel=$n$,
%                ylabel=$\alpha$,
%                colormap/blackwhite,
%        ]
%		    \addplot3[surf] file {PhaseTransvsn_gaussian.dat};
%        \end{groupplot}
%\end{tikzpicture}
%\end{center}
%\caption{The left plot shows the phase transition for the bounded noise case. In this case we observe that since the dependence is logarithmic in $n$ the variance cannot be observed even though we increase $n$. The right plot shows, however, that when the data and noise are both Gaussian, there is a nearly linear dependence on $n$.}
%\label{fig:phasetransvsn}
%\end{figure}			


\begin{figure}[t!]
\begin{center}
\begin{tikzpicture}
        % load the style created in the preamble
	\begin{axis}[
		xlabel={$\alpha$},
		ylabel={SE},
		my legend style compare,
		legend entries={
            	Average SE,
            	Max. SE,
            	Predicted SE bound,
            },
			legend style={
                at={([yshift=-10pt].41,1.01)},
                anchor=north west,
            },
            ymode=log,
            xmode=log,
            enlargelimits=false,
		] 
	
	\addplot table[x index = {0}, y index = {1}]{\SEbound};
	\addplot table[x index = {0}, y index = {2}]{\SEbound};
	\addplot table[x index = {0}, y index = {3}]{\SEbound};

    \end{axis}
\end{tikzpicture}
\end{center}
\end{figure}			


\end{document}


	        \addplot (10,10);
			\addplot table[x index = {0}, y index = {12}]{\PhatPdata};
			\node [text width=1em,anchor=north] at (rel axis cs: 0,1) {\subcaption{\label{fig:prelimfiga}}};
        \nextgroupplot[
        	ymode=log,
            xlabel={$t$},
            title={(b) $\zz = 10^{-2}$},
            ymin=1e-3, ymax=1.01
            ]        ]
        	\addplot table[x index = {0}, y index = {1}]{\PhatPdata};
	        \addplot table[x index = {0}, y index = {2}]{\PhatPdata};
	        \addplot table[x index = {0}, y index = {3}]{\PhatPdata};
	        \addplot (10,10);
			\addplot table[x index = {0}, y index = {4}]{\PhatPdata};
\node [text width=1em,anchor=north] at (rel axis cs: 0,1) {\subcaption{\label{fig:prelimfigb}}};
    \end{groupplot}
    
    
    %          \begin{axis}[view={0}{90},
%               xlabel=$n$,
%               ylabel=$\alpha$,
%               colormap/blackwhite,
%               title=Phase Transition versus $n$ bounded]
%    \addplot3[surf] file {PhaseTransvsn_bounded.dat};
%  \end{axis}
        
